\begin{frame}
    \frametitle{Enquadramento}
    \begin{itemize} 
        \item As redes de acesso sao constituidos por um numero tipicamente
        elevado de equipamentos.
        \item Para gerir e monitorizar um numero tao elevado de equipamentos sao
        necessarias plataformas centralizadas, os \textit{NMS}.
        \item A \textbf{AlticeLabs} desenvolveu a sua propria solucao, no
        contexto dos \textit{NMS}, o \textit{AGORA}. 
        \item Para que todos os equipamentos possam ser geridos e monitorizados
        por este tipo de ferramentas tem de ser configurados. 
    \end{itemize}
\end{frame}

\begin{frame}
    \frametitle{Enquadramento}
    \begin{itemize}
        \item O elevado numero em que estes aparelhos normalmente se apresentam
        faz com que a automatizacao do processo de configuracao dos mesmos seja
        apetecivel.
        \item Neste contexto, foi sugerida a investigacao e desenvolvimento de
        um modelo de automatizacao deste tipo de configuracoes no \textit{AGORA}
        por parte da \textbf{AlticeLabs}. 
        \item O facto de ja existir o \textit{ZTP} de \textit{OLTs} e este
        apresentar grande popularidade levou a que se considerasse a criacao de
        uma solucao de funcoes analogas para os \textit{ONTs}. 
    \end{itemize}
\end{frame}

\begin{frame}
    \frametitle{\checkmark \hspace{1mm} Objetivos}
    \begin{itemize}
        \item Desenho, em tracos gerais, de uma arquitetura para o novo servico
        que satisfaca todas as suas exigencias em termos de funcionalidade. 
        \item Levantamento do estado de arte: relativamente a conceitos
        arquiteturais, ferramentas para a implementacao de microsservicos e
        ferramentas para a automatizacao de tarefas.
        \item Desenvolvimento de algumas alternativas de arquiteturas para o
        modulo \textit{ZTP ONUs}.
        \item Desenvolvimento de um cenario que permita, de forma objetiva,
        comparar estas alternativas. 
        \item Analise de preocupacoes suplementares como seguranca e
        monitorizacao e apresentacao de proposta de melhoramento da arquitetura
        que tenham em conta os conhecimentos adquiridos. 
    \end{itemize}
\end{frame}
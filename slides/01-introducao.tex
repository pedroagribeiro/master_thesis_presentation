\begin{frame}
    \frametitle{Enquadramento}
    \begin{itemize} 
        \item As redes de acesso são constituídas por um número tipicamente
        elevado de equipamentos.
        \item Para gerir e monitorizar um número tão elevado de equipamentos são
        necessárias plataformas centralizadas, os \textit{NMS}.
        \item A \textbf{Altice Labs} desenvolveu a sua própria solução, no
        contexto dos \textit{NMS}, o \textit{\textbf{AGORA}}. 
        \item Para que todos os equipamentos possam ser geridos e monitorizados
        por este tipo de ferramentas têm de ser configurados. 
    \end{itemize}
\end{frame}

\begin{frame}
    \frametitle{Enquadramento}
    \begin{itemize}
        % \item O elevado número em que estes aparelhos normalmente se apresentam
        % faz com que a automatização do processo de configuração dos mesmos seja
        % apetecível.
        \item Estes equipamentos apresentam-se em número muito elevados.   
        % \item Neste contexto, foi sugerida a investigação e desenvolvimento de
        % um módulo de automatizacao deste tipo de configurações no
        % \textit{\textbf{AGORA}} por parte da \textbf{AlticeLabs}. 
        \item Consequentemente, surge a ideia de automatizar o processo de
        configuração e aprovisionamento dos mesmos.
        % \item O facto de já existir o \textit{ZTP} de \textit{OLTs} e este
        % apresentar grande popularidade levou a que se considerasse a criação de
        % uma solução de funções análogas para os \textit{ONTs}. 
        \item A existência de um módulo análogo para \textit{OLT's} deu ainda
        mais tração a esta ideia.
    \end{itemize}
\end{frame}

\begin{frame}
    \frametitle{\checkmark \hspace{1mm} Objetivos}
    \begin{itemize}
        % \item Desenho, em traços gerais, de uma arquitetura para o novo serviço
        % que satisfaça todas as suas exigências em termos de funcionalidade. 
        \item Desenvolvimento de uma arquitetura para um serviço que implemente
        o aprovisionamento automático de \textit{ONU's}.
        % \item Levantamento do estado de arte: relativamente a conceitos
        % arquiteturais, ferramentas para a implementação de microsserviços e
        % ferramentas para a automatização de tarefas.
        \item Levantamento do estado de arte relativamente a arquitetura de
        \textit{software} e ferramentas que poderão ser utilizados no
        desenvolvimento do serviço.
        \item Desenvolvimento de algumas alternativas de arquiteturas para o
        módulo \textit{ZTP ONUs}.
        \item Desenvolvimento de um cenário que permita, de forma objetiva,
        comparar estas alternativas. 
        \item Análise de preocupações suplementares como segurança,
        monitorização e apresentação de propostas de melhoramento da arquitetura
        tendo em conta os conhecimentos adquiridos. 
    \end{itemize}
\end{frame}